\chapter{FEniCS}

The Finite Element library used in the execution of this Master Thesis is from  the FEniCS Project\cite{FenicsWebsite}. FEniCS Project  automates the solution of mathematical models based on differential equations \cite{fenics_book}.

The FEniCS Project started in 2003 with an idea to automate the solution of mathematical models based on differential equations. FEniCS has interfaces for both in C++ and in Python.\cite{fenics_book}

The FEniCS libraries used in the execution of this master Thesis is the DOLFIN interface and the FFC compiler. For problems related to that of Solid Mechanics, an extensions of the FEniCS project had to be used. This is mostly due to the requirement of being able to access the yield function, at each and every gauss point. Also, at each gauss point the plastic tangent tensor has to be calculated. This operation may be achieved by using the FEniCS Solid Mechanics library which is an extension to the FEniCS project.

The FEniCS Solid Mechanics library mentioned above was developed by Kristian B. Ølgaard and Garth N. Wells. The mainline of the library was actively developed between 2013 to 2017 but development since has been slow. Much of the newer developments have been on other branches but not merged to the mainline branch. It may be so because they are not "production ready" or that the original developers of the library are not available for maintenance. 

On investigating the capabilities of the FEniCS Solid Mechanics library, it was found that the library was designed to work for simpler plasticity like $J2$ models. Hence the most complex part of the Thesis was to extend the library capabilities well enough, such that it is possible to execute more complex plasticity models such as Sinfonietta Classica without breaking backward compatibility to already implemented simpler models such as $J2$. More details on how and what are the changes made to the library has been mentioned in detail in the "Implementation" Chapter. In the rest of this chapter a very brief introduction to the FEniCS project has been provided.

\section*{DOLFIN}
DOLFIN is a C++/Python library that functions as the main user interface of FEniCS.
DOLFIN has dual function in the FEniCS project. It is a core component of the FEniCS project. In addition, DOLFIN also acts as the user interface of FEniCS. All communication between a user program, other core components of FEniCS and external software is routed through wrapper layers that are implemented as part of the DOLFIN user interface. When using the C++ interface, the variational forms are expressed in UFL form language.\cite{fenics_book}\\
 
 In Finite element codes, many parts of the Finite Element code can be reused. These may be parts such as the assembly of local matrices to global matrices, mesh information or matrix manipulation. The issue that arises is generally with the formation of the local contribution or the local stiffness matrix per element. This local stiffness matrix is a result of the weak form of the given problem. The weak form is depended on the mathematical model that is being solved. This part generally has to be programmed by the application developer and is unique to every different mathematical model. The FEniCS Project attempts to generalize this part of the algorithm by introducing a Python based language, the "Unified Form Language" or in abbreviated form, the UFL. \cite{ufc_manual}
 
 For the C++ interface, this UFL file generated by the application developer is converted to a low level UFC code using the "Fenics Form Compiler" (FFC) that can be imported into a C++ file. Then using the DOLFIN interface, the global stiffness matrices may be assembled.